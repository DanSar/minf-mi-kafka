\documentclass[12pt]{beamer}
\usepackage[utf8]{inputenc}
\usepackage[ngerman]{babel}
\usepackage{graphicx}

%\usetheme{Dresden}
\usetheme{Boadilla}
%\usecolortheme{dolphin}

\setbeamertemplate{mini frames}{}

\usepackage{tikz}
\usepackage{mwe}
\usepackage{pdfpcnotes}
\usepackage{listings}
\usepackage{xcolor}

\usepackage{hyperref}

\usepackage[%
backend=bibtex
,language=ngerman    % default: autobib, funktioniert manchmal nicht
%,style=authoryear-comp
,style=numeric-comp
%,citestyle=authoryear-comp
,citestyle=numeric-comp
,sorting=none        % sorting: none, as cited
,block=none
,indexing=false
,citereset=none
,isbn=false
,url=false
,doi=false           % prints doi
,natbib=true         % if you need natbib functions
]{biblatex}

\addbibresource{references.bib}

\begin{document}
	\author{Daniel, Fabian, Hauke und Tom}
	\title{Apache Kafka}
	%\subtitle{}
	%\logo{}
	\institute[MI]{Modellierung von Informationssystemen\\ Department Informatik\\ HAW Hamburg}
	\date{01. Dezember 2017}
	%\subject{}
	%\setbeamercovered{transparent}
	\setbeamertemplate{navigation symbols}{}
	
	\setbeamertemplate{caption}[numbered]
	\numberwithin{figure}{section}
	
	
\usebackgroundtemplate{		% declare background template 
	\begin{tikzpicture}[remember picture, overlay]
	\node[opacity=1.0, xshift=-3.8cm, yshift=2.25cm, at=(current page.south)] {
		\includegraphics[scale=0.08]{figure/kafka_diagram.png}};  
	\end{tikzpicture}
}

\begin{frame}[plain]
	\maketitle
\end{frame}

\usebackgroundtemplate{ }    % undeclare background template

\begin{frame}
\tableofcontents
\end{frame}

\begin{frame}
\section{Konzept}
\frametitle{Was ist Apache Kafka?}
\begin{columns}[T]
	\begin{column}[T]{0.49\textwidth}
		
	\end{column}
	\begin{column}[T]{0.49\textwidth}
		
\end{column}
\end{columns}

\centering
\begin{figure}[h]
	\includegraphics[scale=0.1]{figure/kafka_diagram.png}
\end{figure}

Apache Kafka ist eine verteilte skalierbare Streaming Plattform.
\end{frame}


\begin{frame}
\frametitle{Eigenschaften}
Kafka ...
\begin{itemize}
	\item ist ein Message Queuing System
	\item kann Nachrichten speichern
	\item kann Nachrichten verarbeiten
	\item kann all das in Echtzeit
\end{itemize}
\end{frame}

\begin{frame}
\frametitle{Unternehmen und Use Cases}
\begin{itemize}
	\item \includegraphics[scale=0.2]{figure/linkedin_logo.pdf} 
		- Operationsmetriken
	\item \includegraphics[scale=0.2]{figure/cisco_logo.pdf} 
		- OpenSOC (Security Operations Center)
	\item \includegraphics[scale=0.08]{figure/netflix_logo.pdf} 
		- Real-time monitoring and event-processing pipeline
	\item \includegraphics[scale=0.2]{figure/spotify_logo.pdf} 
		- Log Delivery System
	\item \includegraphics[scale=0.1]{figure/twitter_logo.pdf} 
		- Teil der Storm Stream Processing Infrastruktur
\end{itemize}
\end{frame}

\begin{frame}
\frametitle{Komponenten}
	\centering
	\includegraphics[scale=1.4]{figure/kafka-apis.png}
\end{frame}

\begin{frame}
\frametitle{Topics}
	\begin{center}
		B I L D
	\end{center}
\end{frame}

\begin{frame}
\frametitle{Kafka Topics}
\begin{itemize}
	\item Multi-Subscribe ($0$ bis $n$ Consumer)
	\item Kein Push-System
	\item Records in Topics werden persistent gehalten
	\item Topics benötigen eine Cleanup-Policy
		\begin{itemize}
			\item Retention-Time
			\item Retention-Size
			\item Log-Compaction
		\end{itemize}
	\item Topics besitzen Partitionen (partition log)
\end{itemize}
\end{frame}

\begin{frame}
\frametitle{Partitionen}

\end{frame}

\begin{frame}
\frametitle{Kafka Cluster}

\end{frame}
\lstdefinestyle{bashstyle}{
    language=bash,
    backgroundcolor = \color{black},
	basicstyle = \color{white}\footnotesize\ttfamily,
	breakatwhitespace=false,         
    breaklines=true,                 
    captionpos=b,                    
    keepspaces=true,                 
    showspaces=false,                
    showstringspaces=false,
    showtabs=false,                  
    tabsize=2,        
    framexleftmargin=6pt,
    framexrightmargin=6pt,
    framextopmargin=6pt,
    framexbottommargin=6pt, 
    frame=tb, 
    framerule=0pt
}
\lstset{style=bashstyle}
\begin{frame}[fragile]
\section{Tutorial}
\subsection{Quickstart}
\frametitle{Tutorial}
\begin{itemize}
\item Download Kafka~\cite{KafkaDownload}
\end{itemize}
\begin{lstlisting}[]
$ tar -xzf kafka_2.11-1.0.0.tgz
$ cd kafka_2.11-1.0.0
\end{lstlisting}

\begin{lstlisting}[]
$ bin/zookeeper-server-start.sh config/zookeeper.properties
$ bin/kafka-server-start.sh config/server.properties
\end{lstlisting}
\end{frame}

\begin{frame}[fragile]
\frametitle{Tutorial}
\begin{lstlisting}[]
$ bin/kafka-topics.sh --create --zookeeper localhost:2181 --replication-factor 1 --partitions 1 --topic test
$ bin/kafka-console-producer.sh --broker-list localhost:9092 --topic test
> This is a message
> This is another message
\end{lstlisting}

\begin{lstlisting}[]
$ bin/kafka-console-consumer.sh --bootstrap-server localhost:9092 --topic test --from-beginning
> This is a message
> This is another message
\end{lstlisting}
\end{frame}

\begin{frame}[fragile]
\subsection{Properties}
\frametitle{Tutorial}

\begin{table}[h!]
\centering
\begin{tabular}{ |c|c|c| } 
\hline 
Name & Beschreibung & Typ \\ \hline \hline
broker-list/bootstrap.servers & Host und Port für das Cluster & Liste \\ \hline
topic & cell5 & cell6 \\ 
cell7 & cell8 & cell9 \\ 
\hline
\end{tabular}
\caption{Producer Properties}
\label{producer_prop}
\end{table}

\end{frame}

\begin{frame}[fragile]
\frametitle{Tutorial}
\begin{itemize}
\item Broker Properties~\cite{KafkaPropBroker}
\end{itemize}
\end{frame}

\begin{frame}[fragile]
\subsection{Kafka Clients}
\frametitle{Tutorial}
\begin{itemize}
\item Diverse Clients vorhanden~\cite{KafkaClients}
\begin{itemize}
\item Java, Python, Go, C/C++, .NET, Ruby, ...
\end{itemize}
\item Kafka in Java, daher der meiste Support

\end{itemize}
\end{frame}

\begin{frame}[fragile]
\subsection{Twitter App}
\frametitle{Tutorial}
\begin{figure}
\centering
\includegraphics[width=0.95\textwidth]{figure/kafka_tut_diagram.pdf}
\caption{Architektur Twitter App}
\end{figure}
\end{frame}

\definecolor{codegreen}{rgb}{0,0.6,0}
\definecolor{codegray}{rgb}{0.5,0.5,0.5}
\definecolor{codepurple}{rgb}{0.58,0,0.82}
\definecolor{backcolour}{rgb}{0.95,0.95,0.92}
 
\lstdefinestyle{pythonstyle}{
	language=Python,
    basicstyle=\scriptsize\ttfamily,
	backgroundcolor=\color{backcolour},   
    commentstyle=\color{codegreen},
    keywordstyle=\color{magenta},
    numberstyle=\tiny\color{codegray},
    stringstyle=\color{codepurple},
    breakatwhitespace=false,         
    breaklines=true,                 
    captionpos=b,                    
    keepspaces=true,                 
    showspaces=false,                
    showstringspaces=false,
    showtabs=false,                  
    tabsize=2,        
    framexleftmargin=6pt,
    framexrightmargin=6pt,
    framextopmargin=6pt,
    framexbottommargin=6pt, 
    frame=tb, 
    framerule=0pt
}
\lstset{style=pythonstyle}
\begin{frame}[fragile]
\frametitle{Tutorial}
\lstinputlisting[
basicstyle=\scriptsize\ttfamily,
caption=Python Producer]{snippets/snippet-producer.py}
\end{frame}

\begin{frame}[fragile]
\frametitle{Tutorial}
\lstinputlisting[
caption=Python Processor]{snippets/snippet-processor.py}
\end{frame}

\begin{frame}[fragile]
\frametitle{Tutorial}
\lstinputlisting[
caption=Python Consumer]{snippets/snippet-consumer.py}
\end{frame}

\begin{frame}[fragile]
\frametitle{Tutorial}
\begin{center}
\Large{Screencast Demo}
\end{center}
\end{frame}
% \include{section/...}

\begin{frame}[allowframebreaks]
	\title{Literatur}
	\nocite{*}
	\printbibliography
\end{frame}

\end{document}

