\lstdefinestyle{bashstyle}{
    language=bash,
    backgroundcolor = \color{black},
	basicstyle = \color{white}\footnotesize\ttfamily,
	breakatwhitespace=false,         
    breaklines=true,                 
    captionpos=b,                    
    keepspaces=true,                 
    showspaces=false,                
    showstringspaces=false,
    showtabs=false,                  
    tabsize=2,        
    framexleftmargin=6pt,
    framexrightmargin=6pt,
    framextopmargin=6pt,
    framexbottommargin=6pt, 
    frame=tb, 
    framerule=0pt
}
\lstset{style=bashstyle}
\begin{frame}[fragile]
\section{Tutorial}
\subsection{Quickstart}
\frametitle{Tutorial}
\begin{itemize}
\item Download Kafka~\cite{KafkaDownload}
\end{itemize}
\begin{lstlisting}[]
$ tar -xzf kafka_2.11-1.0.0.tgz
$ cd kafka_2.11-1.0.0
\end{lstlisting}

\begin{lstlisting}[]
$ bin/zookeeper-server-start.sh config/zookeeper.properties
$ bin/kafka-server-start.sh config/server.properties
\end{lstlisting}
\end{frame}

\begin{frame}[fragile]
\frametitle{Tutorial}
\begin{lstlisting}[]
$ bin/kafka-topics.sh --create --zookeeper localhost:2181 --replication-factor 1 --partitions 1 --topic test
$ bin/kafka-console-producer.sh --broker-list localhost:9092 --topic test
> This is a message
> This is another message
\end{lstlisting}

\begin{lstlisting}[]
$ bin/kafka-console-consumer.sh --bootstrap-server localhost:9092 --topic test --from-beginning
> This is a message
> This is another message
\end{lstlisting}
\end{frame}

\begin{frame}[fragile]
\subsection{Properties}
\frametitle{Tutorial}

\begin{table}[h!]
\centering
\begin{tabular}{ |c|c|c| } 
\hline 
Name & Beschreibung & Typ \\ \hline \hline
broker-list/bootstrap.servers & Host und Port für das Cluster & Liste \\ \hline
topic & cell5 & cell6 \\ 
cell7 & cell8 & cell9 \\ 
\hline
\end{tabular}
\caption{Producer Properties}
\label{producer_prop}
\end{table}

\end{frame}

\begin{frame}[fragile]
\frametitle{Tutorial}
\begin{itemize}
\item Broker Properties~\cite{KafkaPropBroker}
\end{itemize}
\end{frame}

\begin{frame}[fragile]
\subsection{Kafka Clients}
\frametitle{Tutorial}
\begin{itemize}
\item Diverse Clients vorhanden~\cite{KafkaClients}
\begin{itemize}
\item Java, Python, Go, C/C++, .NET, Ruby, ...
\end{itemize}
\item Kafka in Java, daher der meiste Support

\end{itemize}
\end{frame}

\begin{frame}[fragile]
\subsection{Twitter App}
\frametitle{Tutorial}
\begin{figure}
\centering
\includegraphics[width=0.95\textwidth]{figure/kafka_tut_diagram.pdf}
\caption{Architektur Twitter App}
\end{figure}
\end{frame}

\definecolor{codegreen}{rgb}{0,0.6,0}
\definecolor{codegray}{rgb}{0.5,0.5,0.5}
\definecolor{codepurple}{rgb}{0.58,0,0.82}
\definecolor{backcolour}{rgb}{0.95,0.95,0.92}
 
\lstdefinestyle{pythonstyle}{
	language=Python,
    basicstyle=\scriptsize\ttfamily,
	backgroundcolor=\color{backcolour},   
    commentstyle=\color{codegreen},
    keywordstyle=\color{magenta},
    numberstyle=\tiny\color{codegray},
    stringstyle=\color{codepurple},
    breakatwhitespace=false,         
    breaklines=true,                 
    captionpos=b,                    
    keepspaces=true,                 
    showspaces=false,                
    showstringspaces=false,
    showtabs=false,                  
    tabsize=2,        
    framexleftmargin=6pt,
    framexrightmargin=6pt,
    framextopmargin=6pt,
    framexbottommargin=6pt, 
    frame=tb, 
    framerule=0pt
}
\lstset{style=pythonstyle}
\begin{frame}[fragile]
\frametitle{Tutorial}
\lstinputlisting[
basicstyle=\scriptsize\ttfamily,
caption=Python Producer]{snippets/snippet-producer.py}
\end{frame}

\begin{frame}[fragile]
\frametitle{Tutorial}
\lstinputlisting[
caption=Python Processor]{snippets/snippet-processor.py}
\end{frame}

\begin{frame}[fragile]
\frametitle{Tutorial}
\lstinputlisting[
caption=Python Consumer]{snippets/snippet-consumer.py}
\end{frame}

\begin{frame}[fragile]
\frametitle{Tutorial}
\begin{center}
\Large{Screencast Demo}
\end{center}
\end{frame}